\documentclass{article}
\usepackage[utf8]{inputenc}
\usepackage{fullpage}
\usepackage{comment}

\renewcommand{\baselinestretch}{1.3} 

\title{
	\textbf{\Huge{myTaxiService}}
	\\ 
	\huge{Requirement Analysis and Specification Document}
}
\author{
	Monica Magoni 854091\\Alberto Cibari 852689
}
\date{October 2015}

\begin{document}
	
	\begin{comment}
	Pagina Titolo
	\end{comment}
	\maketitle
	
	
	\begin{comment}
	Pagina Indici
	\end{comment}
	\newpage
	\addtocontents{toc}{~\hfill\textbf{Page}\par}
	\renewcommand*\contentsname{\Huge{Summary}}
	\tableofcontents
	
	
	\begin{comment}
	Pagine Testo 
	\end{comment}
	\newpage
	\section{Introduction}
	
	\subsection{Purpose}
	This document represents the Requirements Analysis and Specification Document (RASD).
	Its aim is to capture all the functional and non-functional requirements that the system-to-be has to respect, in order to satisfy the stakeholders goals, under certain domain properties. This document is also a valid basis for system testing, verification and validation.
	
	\subsection{Scope}
	The aim of our project is to optimize the taxi service of a large city. 
	Passengers, once registered to the application, will have the possibility to request a taxi through either the web application or the mobile application. When a request from a passenger arrives, the system-to-be will be able to look for an available taxi (through a fair management of the taxi queue) and send the code of the incoming taxi and the waiting time to the passenger. Taxi drivers will inform the system of their position and confirm a certain call thanks to a mobile application.\\
	If the first taxi of the queue is not available, the system will forward the request to the second taxi of the queue and put the first taxi in the last position of the queue. This process will continue until it finds an available taxi, in order to satisfy the passenger's request.\\\\
	The peculiar feature of our system is that it will also offer an additional service: it will allow passengers to reserve a taxi in advance, by specifying the origin and the destination of their journey. The reservation must occur at least two hours before the ride and the system will inform the taxi driver ten minutes before the scheduled time.
	
	\subsection{Definitions, acronyms, abbreviations}
	In order to avoid confusion, we want to specify the definition of some words that will be often used in our documentation of the project. 
	\begin{itemize}
		\item PASSENGER: for passenger we mean a person, already registered in the system, who has requested or reserved a taxi either through the web or the mobile applications.
		\item USER: a person who is already registered in the system that can use the application to request or reserve a taxi.
		\item GUEST:  a person who is not registered in the system.
		\item REQUEST: message sent by the user's application to the system in order to require a taxi or make a reservation.
		\item CALL: a task received by the taxi drivers after a user made a request.
		\item TAXI ZONE: for taxi zone we mean a zone that is approximately 2km^2. Each taxi zone has a queue of taxi associated. 
	    \item QUEUE: for queue we mean the list of taxi available at a specific moment in a certain zone.
		\item NOTIFICATION: 
		\item RESERVATION: the action of the user to book a taxi followed by a request to the system.
	
	\end{itemize}

	
	\subsection{Reference documents}
	\subsection{Overview}
	
	
	\section{Actors Identifying}
	The actors of our informative system are:
	\begin{itemize}
	    \item \textbf{Guest}: a person who is not registered in the system. He can not use the features of the system until he registers. 
		\item \textbf{User}: a person who is already registered in the system that uses the application to request or reserve a taxi.
		\item \textbf{Taxi Driver}: a person registered in the system that uses the application to inform the system of the call they take care of and to notify their availability.
		\item \textbf{Administrators}: managers of the web and mobile applications. They manage internally the system and try to fix errors in case of system fault.
	\end{itemize}
	
	\section{Overall Description}
	\subsection{Product perspective}
	\subsection{Product functions}
	\subsection{User characteristics}
	\subsection{Constraints}
	\subsection{Assumptions and Dependencies}
	\section{Specific Requirements}
	
\end{document}