Our group has to inspect two methods of the class \textit{AdminConsoleAuthModule}: the \textit{initialize} and the \textit{validateRequest} methods.\\
This class can be found in the class file under \textit{appserver/admingui/common/src/main/java/org/glassfish/admingui/common/security/\\AdminConsoleAuthModule.java}

This class is responsible for providing the Authentication support needed by the admin console to both access the admin console pages as well as invoke REST requests.

\subsubsection{\textit{initialize} method}
This method configures this authentication module and makes sure all the information needed to continue is present.

The function sets the callback handler of the class.
It also checks if some options are passed. If there are options, it checks that these are not \textit{null} and sets some options for the class, otherwise throws an exception.

\subsubsection{\textit{validateRequest} method}
This method checks if a request is valid or not. If the request is not mandatory or the security check is not needed, the function returns always a 'SUCCESS' on  the validation, otherwise it makes some checks, for example it checks if the Subject is marked as SAVED or if the username or password are null, and returns a 'SUCCESS' or a 'FAILURE'. The last part of the method is used to make a REST request: then, if the response is successful, the RESt token, the Subject and the username are saved.