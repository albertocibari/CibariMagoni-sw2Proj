In this section, we list all the issues we have found in the code, referring to the assignment document. The order in which they are listed is the same as in the document we referred to. If a point is not present in our list is because we have not found problems concerning it.

\subsection{General class problems}
\label{ref-general}

\begin{itemize}
    \item[7.] 
        In the class, when the constant attributes are declared, there is the constant 'logger' (\textit{private static final Logger logger = GuiUtil.getLogger();}) that should be declared using all uppercase.
    \item[13.] 
        In the whole document there are lines of code that are more than 80 characters long. 
        \newline
        Here we list the lines that are not in the methods we have to analyse:
            \begin{itemize}
                \item Line \textit{100} : declaration of \textit{SUPPORTED\_MESSAGE\_TYPES} attribute
                \item Line \textit{326} : declaration of \textit{secureResponse} method
                \item Line \textit{334} : declaration of \textit{cleanSubject} method
            \end{itemize}
    \item[14.] 
        We list the lines that are more than 120 characters long that are present in the previous point of the list: line \textit{100}
    \item[18.] 
        Comments on code lines are adequate for describing the code. Javadoc comments, and class comments in general are poor written or absent. In general the code is very less commented.
    \item[19.] 
        Line \textit{90}: variable declaration; this variable declaration is commented out but there is neither the reason why it has been commented out nor the date.
    \item[23.] 
        The JavaDoc is not complete. There is only a brief description of the class attributes and methods. Also, in the methods, the parameters and the return values are not described by the JavaDoc.
    \item[25.] 
        The static variables are declared after the instance variables. Also the static ones are declared in a non specific order. In this case \textit{private} - \textit{public} - \textit{private}. 
    \item[27.] 
        The class is formed of five methods. The method \textit{cleanSubject} is empty and has a comment \textit{//FIXME} inside. The relevant method in the class is \textit{validateRequest} and is a very long method.
\end{itemize}
\subsection{\textit{initialize} method}

\begin{itemize}
        \item[13.] In the whole document there are lines of code that are more than 80 characters long. 
        \newline
        Here we list the lines that are in the method analyzed in this section:
            \begin{itemize}
                \item Line \textit{132}: the declaration of the method
                \item Lines \textit{138, 144, 154, 155}: strings concatenation
                \item Lines \textit{147, 149}: method concatenation
            \end{itemize}
    \item[14.] 
    We list the lines that are more than 120 characters long that are also present in the previous point of the list: lines \textit{132, 149, 155}
    \item[15.] Lines \textit{138, 139, 144, 145}: the '+' operator should be put in the previous line, in order to break the line after an operator.
    \item[18.] There are few comments in the code. JavaDoc comments are poor written or absent. In general the code is very less commented.\\
    For the JavaDoc there are not the tags \textit{'@param', '@throw'} to describe the parameters of the function.
    \item[23.] View section \ref{ref-general}, point '23' of the list.
    \item[33.] In this method all the variable declarations and instantiations are done to create the value of a class variable at the end of the method. This behavior is still acceptable for the Java convention.
    \item[35.] This function calls some deprecated methods, in particular the methods \textit{getValue} and \textit{putValue} of the class \textit{StandardSession} and so these calls should be avoided.
\end{itemize}
\subsection{\textit{validateRequest} method}

\begin{itemize}
        \item[1.] 
           There are some variables such as '\textit{rd}',  '\textit{ae}' and '\textit{qs}' that have a name too short, so that it is not easy to understand their meaning. Although they are formed of two characters instead of one, they can be consider throwaway variables: indeed, they have only a temporary use. Their name is formed by the initial characters of their type. In particular '\textit{rd}' stands for '\textit{RequestDispatcher}', '\textit{ae}' for '\textit{AuthException}' and '\textit{qs}' stands for '\textit{QueryString}'.
        \item[5.] 
            On line \textit{443} it is used a function called \textit{basic(String, String)} and its name is not a verb. The syntax of each method is correct: the first letter of each addition word is capitalized.
        \item[13.] 
            In the whole document there are lines of code that are more than 80 characters long. 
            \newline
            Here we list the lines that are in the method analyzed in this section:
                \begin{itemize}
                    \item Line \textit{171}: the declaration of the method;
                    \item Lines \textit{191 to 193}: method concatenation;
                    \item Lines \textit{241, 247, 248, 310}: variable declaration and instantiation;
                    \item Lines \textit{214, 291}: 'if' statement;
                    \item Lines \textit{289}: comment;
                \end{itemize}
        \item[14.] 
            We list the lines that are more than 120 characters long that are also present in the previous point of the list: lines \textit{141, 247}.
        \item[15.] 
            Line \textit{178}: it should be better to put the AND operator (\&\&) in the previous line, in order to break the line after an operator.
        \item[17.] 
            Line \textit{265}: there are four spaces that should be avoided.
        \item[18.] 
            Comments on code lines are adequate for describing the code. JavaDoc comments are poor written or absent. In general the code is very less commented.\\
            For the JavaDoc the tags \textit{'@param', '@throw', '@return'} to describe the parameters of the function are absent.
        \item [19.] 
            Line from \textit{233} to \textit{237} is code commented out. There is the reason why it has been commented out, but it does not contain the date.
        \item[23.] 
            View section \ref{ref-general}, point '23' of the list.
        \item[27.] 
            This is a very long method, more than 150 lines of code, and it should be divided into sub methods.
        \item[33.] 
            In this function the major part of the local variables are declared and instantiated just before a compound statement.
        \item[50.] 
            The code that can cause exceptions is handled by the \textit{try-catch} statement. It is correct but can be improved. In fact the functions can throw more than one type of exception, but these exceptions are handled only by a generic catch statement (\textit{catch (Exception ex)}). It would be better if there were multiple catch statements, one for each exception type.
    \end{itemize}

\begin{comment}
    ----------------------------------------------------------------------------------------------------------------------------------------------------------------------------------------------------------------------------------------------
    \begin{enumerate}
        \item There are some variables, such as "ae, resp, rd,..." that have name too short, so that it is not easy to understand their meaning. Maybe they can be consider throwaway variables.
        \item Okay - all classes: there are no one-character variables
        \item Okay - all class
        \item Okay - all class
        \item In the function "validateRequest", a method is called "basic(...)" which is not a verb
        \item Okay - all class
        \item Okay in methods.
            In the class declaration: \\private static final Logger logger = GuiUtil.getLogger();. 'logger' should be "LOGGER"
        \item Okay - All document
        \item Okay - All document
        \item Okay - All document: the bracing style used in all document is the 'Kernighan and Ritchie' style
        \item Okay - both methods
        \item Okay - all document - are we sure?! -> SURE!
        \item More than 80 chars in a line
                \begin{itemize}
                    \item class document: SUPPORTED\_MESSAGE\_TYPES attribute declaration, public void cleanSubject method declaration, public AuthStatus secureResponse method declaration
                    \item initialize: declaration, line 138 \& 144 (string too long), 147 \& 149 (method concatenation too long), 154 \& 155 string concatenation too long
                    \item validateRequest: declaration, 191-193 (methods concatenation), 214 (if statement too long), 241 (variable declaration + instantiation), 147 (variable declaration + instantiation with concat methods), 288 (variabe declaration + instantiation), 289 (comment), 291 (if statement), 307,308,310
                \end{itemize}
        \item   more than 120 chars in a line
                \begin{itemize}
                    \item class document: SUPPORTED\_MESSAGE\_TYPES attribute declaration
                    \item initialize: declaration, 149, 155
                    \item validateRequest: declaration, 247
                \end{itemize}
        \item In function "validateRequest(...)", line 178: it should be       better to put the AND operator (\&\&) in the previous line.
              \newline
              In function "initialize(...)": 
              \begin{enumerate}
                \item line 138 and 139: the + operator should be put in the previous line.
                \item line 144 and 145: the + operator should be put in the previous line.
            \end{enumerate}
        \item Okay-all classes
        \item In the function "validateRequest(...)", line 265: There are 4 spaces that should be avoided.
        \item Comments on code lines are adequate for describing the code. Javadoc comments, and class comments in general are poor written or absent. In general the code is very less commented.
        \item   \begin{itemize}
                    \item Variables declaration: line 90. This variable declaration is commented out but there is neither the reason why it has been commented out nor the date.
                    \item validateRequest:  line from 233 to 237 is code commented out. There is the reason why it has been commented out, but it does not contain the date.
                \end{itemize}
        \item Okay
        \item Okay
        \item TODO ????????????????????????????????????????????????????????????????????????????????????????????????????????????????????????????????? Che si fa con questo punto?
        \item The javadoc isn't complete. There is only a brief description of the class attributes and methods. Also, in the methods, parameters and the return values are not described by the JavaDoc.
        \item Okay
        \item The static variables are declared after the instance variables. Also the static ones are declared in a non specific order: private -> public -> private. 
        Okay for other parts
        \item Okay
        \item The class is formed by 5 methods. The method \textit{cleanSubject} is empty and has a comment \textit{//FIXME} inside. The relevant method in the class is \textit{validateRequest} and is a very long method
        \item Okay
        \item Okay
        \item Okay
        \item Okay
        \item Okay
        \item
            \begin{itemize}
                \item initialize: in this method all the variables declarations are done just before the return statement, to create the return value. This behavior is still acceptable for the Java convention
                \item validateRequest: In this function the major part of the local variables are declared and instantiated just before a compound statement. The standard Java convention allows for declarations to appear at the beginning of any compound statement, however this is discouraged because it can lead to issues. So this behaviour is not recommended
            \end{itemize}
        \item Okay
        \item The validateRequest function calls some deprecated methods, in particular the methods \textit{getValue} and \textit{putValue} of the class \textit{StandardSession}
        \item Okay ??
        \item No Array
        \item Okay
        \item NoArray
        \item Okay. And also object are checked if they are not null with '=='
        \item Okay
        \item Okay
        \item Okay
        \item Okay
        \item Okay. La maggior parte degli if hanno dentro degli Or, quindi l'ordine dei controlli non conta. Non ci sono calcoli aritmetici. L'unico controllo che ho trovato è sulla riga 154 e 155, ma ci sono le parentesi per dividere le azioni da fare. Corretto tutto quindi.
        \item Okay. Example in row 269
        \item Not Found
        \item Not found
        \item Okay
        \item Okay. Also when a function can throw more than one type of exception, in the function there is only one catch of a generic Exception, and not a multiple catch statement, one for each exception type.
        
        \item Okay. There's a great use of generic types and explicit cast conversions
        \item All the code that can throw an exception is handled
        \item  \begin{itemize}
                \item initialize: (Okay Gestite correttamente) AuthException thrown when error in option parameters
                \item validateRequest: (Okay Gestite correttamente)
                    \begin{itemize}
                        \item MalformedURLException thrown when URL parsed is not right. In the catch handler a new exception is thrown: IllegalArgumentException.
                        \item when the function tries to handle the callback, create a RequestDispatcher or send a response a generic \textit{Exception} can be thrown. In the catch handler for these generic exceptions it is created a AuthException, filled with the info of the generic exception and thrown.
                    \end{itemize}
                
                
            \end{itemize}
        \item no switch
        \item no switch
        \item no loops
        \item no Files
        \item no Files
        \item no Files
        \item no Files
    \end{enumerate}
\end{comment}