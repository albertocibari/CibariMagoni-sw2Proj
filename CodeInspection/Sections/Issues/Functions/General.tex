\subsection{General class problems}
\label{ref-general}

\begin{itemize}
    \item[7.] 
        In the class, when the constant attributes are declared, there is the constant 'logger' (\textit{private static final Logger logger = GuiUtil.getLogger();}) that is declared using all lowercase characters.
    \item[13.] 
        In the whole document there are lines of code that are more than 80 characters long. 
        \newline
        Here we list the lines that are in the class, but not in the methods we have to analyse:
            \begin{itemize}
                \item Line \textit{100} : declaration of \textit{SUPPORTED\_MESSAGE\_TYPES} attribute
                \item Line \textit{326} : declaration of \textit{secureResponse} method
                \item Line \textit{334} : declaration of \textit{cleanSubject} method
            \end{itemize}
    \item[14.] 
        We list the lines that are more than 120 characters long that are present in the previous point of the list: line \textit{100}
    \item[18.] 
        Comments on code lines are adequate for describing the code. Javadoc comments, and class comments in general are poor written or absent. In general the code is very less commented.
    \item[19.] 
        Line \textit{90}: variable declaration; this variable declaration is commented out but there is neither the reason why it has been commented out nor the date.
    \item[22.] 
        Since the class has no meaningful JavaDoc, it is very difficult to proceed with the analysis of the implementation of this class.
    \item[23.] 
        The JavaDoc is not complete. There is only a brief description of the class attributes and methods. Moreover, there are three methods where the Javadoc is completely absent: \textit{secureResponse}, \textit{cleanSubject} and \textit{isMandatory}. Also, in all the methods of this class, the parameters and the return values are not described by the JavaDoc.
    \item[25.] 
        The static variables are declared after the instance variables. Also the static ones are declared in a non specific order. In this case \textit{private} - \textit{public} - \textit{private}. 
    \item[27.] 
        The class is formed of five methods. The method \textit{cleanSubject} is empty and has a comment \textit{//FIXME} inside. The relevant method in the class is \textit{validateRequest} and is a very long method, which should be split into two methods.
\end{itemize}