\subsection{\textit{initialize} method}

\begin{itemize}
        \item[13.] In the whole document there are lines of code that are more than 80 characters long. 
        \newline
        Here we list the lines that are in the method analyzed in this section:
            \begin{itemize}
                \item Line \textit{132}: the declaration of the method
                \item Lines \textit{138, 144, 154, 155}: strings concatenation
                \item Lines \textit{147, 149}: method concatenation
            \end{itemize}
    \item[14.] 
    We list the lines that are more than 120 characters long that are also present in the previous point of the list: lines \textit{132, 149, 155}
    \item[15.] Lines \textit{138, 139, 144, 145}: the '+' operator should be put in the previous line, in order to break the line after an operator.
    \item[18.] There are few comments in the code. JavaDoc comments are poor written or absent. In general the code is very less commented.\\
    For the JavaDoc there are not the tags \textit{'@param', '@throw'} to describe the parameters of the function.
    \item[23.] View section \ref{ref-general}, point '23' of the list.
    \item[33.] In this method all the variable declarations and instantiations are done to create the value of a class variable at the end of the method. This behavior is still acceptable for the Java convention.
    \item[35.] This function calls some deprecated methods, in particular the methods \textit{getValue} and \textit{putValue} of the class \textit{StandardSession}
\end{itemize}