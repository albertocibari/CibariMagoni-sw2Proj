The aim of our project is to optimize the taxi service of a large city. 
Passengers, once registered to the application, will have the possibility to request a taxi through either the web application or the mobile application. When a request from a passenger arrives, the system-to-be will be able to look for an available taxi (through a fair management of the taxi queue) and send to the passenger the code of the incoming taxi and the waiting time. Taxi drivers will inform the system of their availability and confirm a certain call thanks to a mobile application.
\newline
If the first taxi of the queue is not available, the system will forward the request to the second taxi of the queue and put the first taxi in the last position of the queue. This process will continue until the system finds an available taxi, in order to satisfy the passenger's request.
\newline
Our system will also offer some programmatic interfaces for the development of additional functions, such as taxi sharing.
\newline
The system offers interfaces (APIs) to enable the developement of new services.
\newline
One of these services is the 'Taxi Sharing' option.
It will allow passengers to reserve a taxi in advance, by specifying the origin and the destination of their journey. The reservation must occur at least two hours before the ride and the system will inform the taxi driver ten minutes before the scheduled time.