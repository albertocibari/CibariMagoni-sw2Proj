\subsubsection{Guest Registration}
Guests can subscribe to the system via mobile or web application. In both cases they have to fill a form with their personal data and authorize the recipient to use and process their personal details.
If the guest accepts the conditions then he can complete the registration, otherwise it is canceled.
Once a guest is registered, the system checks the consistency of the data inserted by the guest and if everything is correct a mail is sent to the new user. \newline
The registration can also be done by taxi drivers and by a generic user of the service. Taxi drivers must complete the form with more specific data.

	\subparagraph{Functional requirements}
	\noindent
		\begin{itemize}
			\item  During the registration phase guests must choose which kind of account to create: user or taxi driver.
			If the account is a user account then guests must provide:
			\begin{itemize}
				\item Name and surname
				\item Email address
				\item Username
				\item Password
				\item Phone number
			\end{itemize}
			If the account is a taxi driver account, the guest must provide the same information of a user account, with in addiction:
			\begin{itemize}
				\item Taxi Licence ID
			\end{itemize}
			\item The username must be an alphabetic string.
			\item The password must be an alphanumeric string with at least a number and a uppercase character.
			\item The password's length must be between 8 and 16 characters.
			\item There can not be two accounts with the same e-mail and of the same type (user, taxi).
			\item The username is case-insensitive.
			\item Before sending the registration form, the guest will be asked to accept the terms and conditions contract.
		\end{itemize}


\subsubsection{User login/taxi driver login}
Users must be logged in the system to use its services.
During the login process users must choose the type of account (user or taxi), insert the username and the password. If everything is fine, then the users are redirected to their homepage.
In the 'user login' page, guests can be redirected to the 'registration' page.

	\subparagraph{Functional requirements}
	\noindent
		\begin{itemize}
			\item In the 'user field', users can insert the email or the username chosen during the registration phase.
			\item If a user makes a mistake while inserting the credentials, a suggestion field in the page pops up, and it will help the user to recover his credentials.
			\item After inserting the password for 3 times, the account freezes and the user receives an e-mail with the instructions to unfreeze the account.
			\item The 'username field' is case insensitive.
			
		\end{itemize}

\subsubsection{User home page/taxi driver home page}  
After the login process, users are redirected to their main page. Here they have the possibility to choose different actions to perform.
For example, they can reset their password, change personal information or delete their account.
If the user who reached this page is a generic user, he can choose the option of creating a request or a reservation.
If the user is a taxi driver, there are others types of actions available: for instance, he can view the pending requests sent by the system and accept or decline them.

	\subparagraph{Functional requirements}
	\noindent
		\begin{itemize}
			\item A generic user can:
			\begin{itemize}
				\item Change password.
				\item Change account information.
				\item Make a taxi request.
				\item Make a Reservation.
				\item Delete requests or reservation.
				\item Visualize all the requests/reservations done.
				\item View the history of all the rides done.
			\end{itemize}
			\item A taxi driver user can:
			\begin{itemize}
				\item Change password.
				\item Change account information.
				\item Delete account.
				\item View the history of all the rides done.
				\item Activate/Deactivate taxi service.
				\item View the pending request.
				\item Track the taxi via GPS.
			\end{itemize}
			
			\item During the 'password changing' phase, the old password should provided with the new one.
			\begin{itemize}
				\item The new password must be an alphanumeric string with at least a number and a uppercase character.
				\item The new password length must be between 8 and 16 characters. 
				\item The new password must be different from the old one.
			\end{itemize}
			\item During the 'account deletion' phase, after the confirmation of the user, a mail is sent to him with a link that allows to deactivate the account.
			\item The deletion account function delete the account but keeps all the user information. The information are kept as history in the DB.
			\item Activate/Deactivate taxi service: the taxi driver use this function to tell the system to activate/deactivate the taxi service adding/removing him from the taxi queues in the city zones.
			\item A request received by a taxi driver is displayed in his main page for one minute, then it is passed to the next user in the city zone queue.
		\end{itemize}


\subsubsection{Activation/deactivation of the taxi services}
When a taxi driver is visiting his home page in the application, he should be able to activate or deactivate his taxi service sending a notification to the system.
When the system receives the notification from the user application about activating/deactivating the taxi system, it checks if the GPS installed in the car connected to the account is on. If there are no errors the taxi is put into a taxi zone queue/removed from the queue. And the system sends him a notification

	\subparagraph{Functional requirements}
	\noindent
		\begin{itemize}
			\item The activation of the service can occur only if the GPS installed on the car is active, otherwise the taxi driver receives an alert on his application.
			\item If the system can not receive the GPS signal every 5 minutes, it sends a notification to the user.
			\item If the system can not receive the GPS signal for more than 20 minutes, the taxi service is deactivated.
			\item The taxi service can be deactivated only if the user isn't taking a request/reservation.
		\end{itemize} 


\subsubsection{Request making}
The user should be able to make a taxi request from the web application and the mobile application. If the request comes from the web application, the user has to tell the system the origin of the ride, otherwise the origin is taken using the GPS of the mobile application.
After the user makes the request, he receives a notification from the system with the identifier of the taxi and the possible arrival time of the taxi.
Once the request is sent to the system, the system has to send it to the first taxi driver available, who can accept or deny the request.

	\subparagraph{Functional requirements}
	\noindent
		\begin{itemize}
			\item If the request is made using the web application, the system shall be provided with the passenger location inserted directly by the user.
			\item If the request is made using the mobile application, the system can use the GPS position to identify the user location.
			\item If the GPS is not available, the system tells the passenger to insert the correct origin location for the taxi ride.
			\item If the origin is not specified (neither from GPS nor from the user insertion), the user can not proceed to send the request.
			\item If a request is not accepted by a taxi driver, it is passed to the next driver in the queue of the city zone.
		\end{itemize}


\subsubsection{Reservation making}
The user should be able to make a taxi request from the web application and the mobile application. The reservation must contain the origin and the destination of the ride. The user has to specify also the date and hour of the reservation. If the values for the reservation are valid, the user can send the request to the system. Then the user will receive a notification with the taxi identifier.

	\subparagraph{Functional requirements}
	\noindent
		\begin{itemize}
			\item The origin and the destination must be valid [View assumptions]. They must be within 10Km from the city boundaries.
			\item If the GPS is available the user can use the current position as origin of the ride.
			\item The date and the time of the reservation must be valid values.
			\item The system does not allow reservation after 24 hours from the current system time.
			\item The system sends a notification to the user 10 minutes before the hour chosen by the user for the ride. The notification contains the taxi identifier.
		\end{itemize}


\subsubsection{Taxi driver notification handling}
After a taxi driver activates the taxi service, he can receive a notification request for a taxi ride. He can accept or refuse the request. If the request is accepted, he has to go to the origin location indicated in the notification.

    \subparagraph{Functional requirements}
    \noindent
        \begin{itemize}
            \item After the taxi driver receives a notification, if the system does not receive an answer within 1 minute, the notification is passed to the following driver in the taxi queue.
            \item If there are no available taxi drivers in a city zone, the system looks for an available taxi for a maximum of 10 times. 
            \item If the system can not find an available taxi, it sends a notification of failure to the user.
            \item When the taxi reaches the origin zone, the system asks him to confirm the begin of the ride.
            \item When the taxi reaches the destination zone, the system asks him to confirm the end of the ride.
            \item After the taxi driver confirms the begin of a ride, the system sets his status to 'busy'.
            \item After the taxi driver confirms the end of a ride, the system sets his status to 'available'.
            \item When a ride is completed, the system stores the data of the ride into the DB.
        \end{itemize}
        
        
\subsubsection{Zone queue handling}
The system has to manage the queue of taxi in all the different city zones. It has to schedule the requests and send them to the right taxi.
    
	\subparagraph{Functional requirements}
	\noindent
	    \begin{itemize}
	        \item The queues use the FIFO policy.
	        \item When a taxi connects to the system and sets its status to 'available', the system insert it into a queue.
	        \item The queue is identified by the GPS positions of the city zones: taxi are put in the queues according to their GPS location.
	    \end{itemize}
    
