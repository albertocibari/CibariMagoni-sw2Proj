List of tools that will be used to perform integration testing:
\begin{itemize}
<<<<<<< HEAD
    \item Manual testing: there will be an accurate selection of the most crucial functionalities (i.e. functions with exceptional parameters) to be manually tested.
    \item JUnit: although this framework is mainly known for unit tests (indeed we will use it also for unit testing but this is not concerned in this document), it will be also used during the integration phase.
    \item Mockito: this framework will be used to mock stubs and drivers that are needed in the integration tests.
=======
    \item JUnit: although this framework is mainly known for unit tests (indeed we will use it also for unit testing but this is not concerned in this document), it will be also used during the integration phase.
    \item Mockito: this framework will be used to mock stubs and drivers that are needed in the integration tests.
    \item Arquillian: it will be used to check the integration of containers with JavaBeans.
>>>>>>> origin/master
\end{itemize}
Moreover, JMeter will be used to set up tests in order to analyze the performances of the system. Indeed, we think that it is very useful to use this framework to verify if the non-functional requirements of our system (described in the RASD) are satisfied.
\newline
In particular, with JMeter we will see how the server and the database behave under a heavy load and with a great number of virtual users (simulated with thread group) simultaneously connected.