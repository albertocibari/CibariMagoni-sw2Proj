\label{scale-driver}

The COCOMO II model identifies five Scale Drivers:
\begin{itemize}
    \item Precedentedness: reflects the previous experience of the organization with this type of project.
        \newline
        Since we do not have a lot of experience concerning this kind of project, the value of precedentedness of our project is low.
    \item Development flexibility: reflects the degree of flexibility in the development process; for example, very high means that the client set only general goals.
        \newline
        In our project, goals were well defined by the client but not in a very specific way; indeed, they were general. So the development flexibility can be very high.
    \item Architecture/risk resolution: reflects the the extent of risk analysis carried out.
        \newline
       Thanks to session control, security access, and GPS tracking mostly of the risks were eliminated then this value will be high.
    \item Team cohesion: reflects how well the development team know each other and work together.
        \newline
        During the project, each member of the team was assigned to a specific task; we scheduled some meetings in order to overcome the difficulties we could have and to discuss about the main topics of each task. Our strategy was effective since we have been able to accomplish every task in time and accurately.
    \item Process maturity: reflects the process maturity of the organization.
        \newline
        This factor has been evaluated according to the 18 Key Process Area (KPAs) in the SEI Capability Model. The value obtained is nominal.
\end{itemize}
\newpage
The values associated to each Scale Driver are listed in the following table:
\newline
\begin{table}[H]
    \centering
    \begin{tabular}{|l|l|l|}
        \hline
        \textbf{Scale Drivers} & \textbf{Value} & \textbf{Weight}\\
        \hline
        Precedentedness & Low & 4.96\\
        \hline
        Development flexibility & Very High & 1.01\\
        \hline
        Architecture/risk resolution & High & 2.83\\
        \hline
        Team cohesion & High & 2.19\\
        \hline
        Process maturity & Nominal & 4.68\\
        \hline
        Total & \multicolumn{2}{|r}{\manuallabel{SD}{15.67}}\ref{SD} \\
        \hline
    \end{tabular}
\end{table}

To calculate the exponent E in the effort formula, we use the value we found of the scale drivers.
\begin{equation}
    E = B + 0.01 * \ref{SD} = 0.91 + 0.01 * \ref{SD} = \manuallabel{E}{1.066}\ref{E}
\end{equation}
Where B is the scaling base-exponent, and it is given by COCOMO. It's value is 0.91.