In this section we perform the Function Points Analysis; this analysis has the aim to give an estimation of the project size, according to its functionalities. Indeed, we can estimate the LOC (Lines of Code) starting from the Function Points count.
This approach identifies different function types:
\begin{itemize}
    \item Internal Logic File;
    \item External Interface File;
    \item External Input;
    \item External Output;
    \item External Inquiry.
\end{itemize}
The meaning of each function type will be clarified later on.
\newline
Before computing the effort, we have to count the number of operations/data that correspond to a specific Function Points type; a weight is associated to each of the Function Points type, depending on its complexity; the total effort is computed by multiplying each count by the weight and summing all values.
\newpage
The weights we will use to calculate the effort are listed in the following table:
\begin{table}[H]
    \centering
    \begin{tabular}{|1|1|1|1|}
        \hline
        Function Types & \multicolumn{3}{c|}{ Complexity } \\ \cline{2-4} 
    
        & Simple & Medium & Complex \\ \cline{1-3} 
        \hline
        Internal Logic Files & 7 & 10 &  15\\
        \hline
        External Interface Files & 5 & 7 & 10 \\
        \hline
        External Input & 3 & 4 & 6 \\
        \hline
        External Output & 4 & 4 & 7 \\
        \hline
        External Inquiry & 3 & 4 & 6 \\
        \hline
    \end{tabular}
\end{table}


