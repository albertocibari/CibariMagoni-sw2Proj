In this section we talk about the risk this project can face during its lifetime, from the project planning to the release and maintenance of the system.

The risk are divided into 3 groups: 
\begin{itemize}
    \item Project risks.
    \item Technical risks.
    \item Business risks.
\end{itemize}

\subsection{Project risks}
This kind of risks threaten the project plan and can road to a slip of the project schedule and an increase of costs.

\begin{itemize}
    \item Delays. The project could require more time than expected, and go beyond the deadline. If this happens, it is possible to release an incomplete version of the product where the most important features are developed, leaving the less important for a second release
    \item Lack of experience. In fact the team members have no experience in programming in the Java EE framework. This problem can have moderate effects at the beginning of the developing, but once become confident with the framework, its effect are less felt.
    \item Change of requirements. The requirements of the project may change and some sections of the project should be redefined.
    \item Team misunderstandings. The project requirements can misinterpreted due to a lack of communication. 
\end{itemize}

\begin{table}[h]
\centering
    \begin{tabular}{|l|l|l|}
        \hline
        Risk & Probability & Effects \\
        \hline
        Delays & High & Moderate \\
        \hline
        Lack of experience & Very High & Moderate \\
        \hline
        Change of requirements & High & Moderate \\
        \hline
        Team misunderstandings & Low & Moderate \\
        \hline
    \end{tabular}
\end{table}


\subsection{Technical risks}
This kind of risks threaten the quality and timeliness of the software to be produced.

\begin{itemize}
    \item Architecture lacks flexibility. The architecture is incapable of supporting change requests and needs to be reworked.
    \item Scalability. Components can't be scaled to meet performance demands. 
    \item Vulnerability. Technology components have security vulnerabilities due to software bugs. This can expose personal data of the users.
    \item Stability. Some components of the system can crash and lead to data losses.
    \item Project management tool problems \& issues. For example the maintenance of a clean code can lead to bugs in the system. To resolve this problem it is possible to use tools for the code inspection.
\end{itemize}

\begin{table}[h]
\centering
    \begin{tabular}{|l|l|l|}
        \hline
        Risk & Probability & Effects \\
        \hline
        Flexibility & Moderate & Serious \\
        \hline
        Scalability & Low & Serious\\
        \hline
        Vulnerability & Moderate & Catastrophic \\
        \hline
        Stability & Moderate & Serious\\
        \hline
        Code issues & Low & Moderate \\
        \hline
    \end{tabular}
\end{table}


\subsection{Business risks}
This kind of risks threaten the viability of the software to be built.

\begin{itemize}
    \item Financial problems. During the development of the project or after its deployment the funds may be insufficient to maintain the project. This problem can be solved doing an accurate analysis before the beginning of the development.
    \item Product incurs legal liability. The product has quality issues that harm the customers. 
\end{itemize}

\begin{table}[h]
\centering
    \begin{tabular}{|l|l|l|}
        \hline
        Risk & Probability & Effects \\
        \hline
        Financial problems & Low & Catastrophic \\
        \hline
        Legal problems & Low & Catastrophics \\
        \hline
    \end{tabular}
\end{table}


